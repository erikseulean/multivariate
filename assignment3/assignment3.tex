\documentclass[11pt]{article}

\usepackage{graphicx}
\usepackage{float}
\usepackage{amsmath}
\usepackage{url}
\usepackage{listings}
\usepackage{enumitem}
\usepackage{caption}
\usepackage{longtable}
\usepackage[a4paper,margin=1in]{geometry}

\renewcommand{\baselinestretch}{1.5}
\everymath{\displaystyle}
\date{}

\title{Clustering majors by socio-economic properties. A guide for students and governments}

\author{Erik-Cristian Seulean, 210001411, Group 5}
\begin{document}
\newpage
\maketitle

\section*{Introduction}
A few days ago, Nadhim Zahawi, the Secretary of State for Education in UK decided to take measures against universities due to concerns regarding employability and sustainability of possible graduates \cite{dailymail}. As society evolves, some degrees become obsolete while others that appear overnight don't offer favorable trajectories, demanding high education costs. For a future student, avoiding these traps might actually be the best thing to do, possibly even more important than having the highest grades possible. 

With this idea in mind, both students and governments would benefit from statistical analysis done on university majors. This analysis would allow clustering university majors based on socio-economic characteristics, and enable both parties to take informed decision on what degree to pursue or what universities to cut from public funding due to low employability of graduates. Ultimately, both students and governments are interested in knowing what groups of majors have the highest unemployment rate and what universities offer the highest median income, in addition to preferable majors for each gender. As a result, the analysis in the following section will focus on answering these questions and offer scientific advice in this regard.

\subsection*{The data}
The data used in this analysis consists of 172 observations and 17 numerical variables. Each observation represents one major, and includes socio-economic properties such as number of female graduates, unemployment rate after graduation, number of jobs that require college education or median income. The source for this data is the American Community Survey and represents majors surveyed between 2010-2012\cite{acs}, and was initially used as part of an article appeared on FiveThirtyEight \cite{fivethirtyeight}, that offered insights into picking college majors based on economic insights.

\section*{Technical details}
For the purpose of this analysis, the two main methods that were used are principal component analysis (PCA) and k-means clustering. Principal component analysis is a method that helps reduce the number of features present in a dataset to a smaller dimension, while keeping as much variability in the data as possible. In general, PCA is useful for datasets that contain a high number of correlated features which can be combined into a subset of features that are uncorrelated, but maintain as much variability of initial features as possible.

\begin{figure} [H]
    \begin{center}
        \includegraphics[scale=0.5, width=10cm]{images/feature_correlation.png}
        \caption{Correlation of features in the dataset}
        \label{fig:featurecorrelation}
    \end{center}
\end{figure}

As we can see in Figure~\ref{fig:featurecorrelation} there are two well-defined correlated groups of features, one containing 15 features that have a high correlation between them, and another one that contains 4 features. The only negative correlation that we can see is between the share of female graduates, and median, 25\% and 75\% quantiles, where higher the number of female graduates relates to lower incomes.

After applying PCA in this scenario, 83\% of the variability in the initial data is represented by 2 components, while 3 components would represent 89\% of variability. In order to pick the number of components visually, the scree plot in Figure~\ref{fig:screeplot} indicates that 2 components could be sufficient. From dimension 3 onwards, the percentage of variance explained is not dropping significantly anymore and these components together represent a small part of the total variability in the dataset.

\begin{figure} [H]
    \begin{center}
        \includegraphics[scale=0.5, width=9.5cm]{images/scree_plot.png}
        \caption{Percentage of variance explained by each dimension}
        \label{fig:screeplot}
    \end{center}
\end{figure}

Naturally, after selecting the first 2 PCAs, the question arises in understanding what these two components represent and how they relate to the initial features in the dataset. In Figure~\ref{fig:featuredirection} the first component, denoted as Dim 1, on the x-axis, seems to be aligned with the first correlated group of features in figure~\ref{fig:featurecorrelation}, the higher the number of women, men, unemployment rate, the higher the value on the x-axis. The second PCA is correlated with the number of female graduates and inverse correlated with the three variables representing the income after university.

\begin{figure} [H]
    \begin{center}
        \includegraphics[scale=0.5, width=10cm]{images/feature_direction.png}
        \caption{First two PCAs and feature direction}
        \label{fig:featuredirection}
    \end{center}
\end{figure}

This is further evidence that the first two components are capable of representing the most important characteristics of the dataset, which were discovered during the initial exploratory analysis (via the correlation plot).

In order to be able to group the majors into clusters, the first two PCAs are used as input to a k-means clustering algorithm. The reason that PCA is applied before using k-means is that it helps reduce the noise in the data and the dimension of the features, which could potentially speed up the clustering algorithm \cite{kmeanspca}.  K-means works by allocating points to a cluster based on the distance to the center of each cluster and allocating the cluster that yields the shortest distance. The algorithm works iteratively, by computing the cluster center and evaluates which cluster the points belong to, steps that are being repeated until no more significant changes occur. The algorithm requires the number of clusters to be specified beforehand or if not known, using visual tools such as silhouette or scree plots can help to find the appropriate number of clusters. In this particular scenario, neither scree plots nor the silhouette method gives satisfactory results, both suggesting that 2 clusters would be the best configuration. An alternative solution was considered in this scenario, using the fact that each major is part of one of the 16 major categories available. For a given number of clusters, from 2 to 16, the purity of each cluster was calculated using a Gini Index~\cite{giniindex}, where a pure cluster represents a cluster that contains only majors part of the same major category. From the 15 possible options for the number of clusters, 9 clusters represent the best mean Gini Index score.

\begin{figure} [H]
    \begin{center}
        \includegraphics[scale=0.5, width=10cm]{images/cluster_configuration.png}
        \caption{9 polygons each representing one cluster}
        \label{fig:clusters}
    \end{center}
\end{figure}

\section*{Results and summaries} 

The socio-economic features presented in the initial dataset only partially recover the 16 major categories existing in the initial dataset and the majority of the clusters are impure. Nevertheless, the results are worth exploring further, given that there are majors that are across major categories that probably should to be part of the same cluster.

\paragraph*{Gini Index 0 - maximum purity}

\begin{center}
    \begin{tabular}{||c c||} 
     \hline
     Major & Major category \\ [0.5ex] 
     \hline\hline
     Petroleum Engineering & Engineering \\ 
     \hline
     Mining and material Engineering & Engineering \\
     \hline
     Metallurgical Engineering & Engineering\\
     \hline
     Nuclear Engineering & Engineering\\
     \hline
    \end{tabular}
    \captionof{table}{Cluster with the highest purity} 
\end{center}



This cluster represents a subset of engineering majors that are probably the closest related to each other even if we don't consider socio-economic features. All the 4 majors are related by the fact that they all deal with highly valuable commodities, the extraction and processing of these commodities. In this regard, the clustering is so good that if somebody would be asked to handpick 4 majors that are related to each other from the entire dataset, there would be a high probability to pick these 4 majors.

\paragraph*{Gini Index 0.44}

\begin{center}
    \begin{tabular}{||c c||} 
     \hline
     Major & Major category \\ [0.5ex] 
     \hline\hline
     Computer science & Computers \& Mathematics \\ 
     \hline
     Mathematics & Computers \& Mathematics \\
     \hline
     Mechanical Engineering & Engineering\\
     \hline
     Electrical Engineering & Engineering\\
     \hline
     General Engineering & Engineering\\
     \hline
     Civil Engineering & Engineering\\
     \hline
    \end{tabular}
    \captionof{table}{Cluster with the highest purity} 
\end{center}

While this cluster has majors coming from two major categories, resulting in a higher Gini Index, the majors are all related to each other. The job prospects between them are generally the same, all requiring a high level of mathematical understanding to be successful. In many situations, somebody studying mechanical engineering or electrical engineering can easily do a postgraduate in computer science or vice versa, and furthermore the job offerings for graduates from one major are likely accessible to candidates from any of the other degrees. As a conclusion, this cluster is expected as well, maybe not as closely related as the previous one, but not far apart either.

\paragraph*{Gini Index 0.62:}As we can see in Figure~\ref{table:three} (Appendix) compared to the other two clusters, this cluster contains 29 majors across 6 different major categories.

% \begin{itemize}[leftmargin=*]
%     \item  \textbf{Engineering:} Naval Architecture, Chemical, Computer, Aerospace, Biomedical, Materials, Mechanics, Biological, Industrial and Manufacturing, Architectural, Electrical, Materials, Miscellaneous, Environmental, Engineering Technologies, Geological and Geophysical, Industrial production, Construction
%     \item \textbf{Computer \& Mathematics:} Computer and information systems, Information sciences, Mathematics and computer science
%     \item \textbf{Business:} Actuarial science, Management Information Systems and Statistics, Operations, Logistics and E-commence
%     \item \textbf{Industrial Arts:} Construction services, Military Technologies 
%     \item \textbf{Law \& Public policy:} Court reporting, Public policy
%     \item \textbf{Physical sciences}: Astronomy and astrophysics
% \end{itemize}

Clearly these majors have much higher dissimilarities in terms of topics than the previous two clusters and contains some majors that are STEM while others that are humanities. While the business majors and the industrial arts ones are somehow related to the engineering or the mathematics ones, the majors from the law and public policy category aren't related by area of study. In this case, socio-economic factors are the ones that drive the clustering of unrelated majors together. The share of women in this cluster is below 50\% for almost all degrees, while the mean income is the second highest among all clusters.  

The rest of the clusters have a Gini Index ranging between 0.78 and 0.9 and having between 2 and 48 majors. These are presented in the Appendix.

\paragraph*{Which cluster has the highest median income ?}
The cluster that contains the 4 engineering degrees: petroleum, mining and materials, metallurgical and nuclear engineering has a median income of \$80,000, which is \$25,000 more than the second-highest cluster, but it also has the lowest percentage of female graduates, with only 13\% and also the highest unemployment rate of 8\%. This is definitely a high-risk high reward degree, where you can have a very good wage early in your career but also have higher chance than other majors to be unemployed. The median income of this cluster twice the median of all 172 majors available in the dataset.

\paragraph*{Which cluster has the highest percentage of female graduates ?}

Cluster 8 in Table~\ref{table:eight} represents the cluster with the highest percentage of female graduates, 71\%. This cluster does not contain any science related degrees and also has the lowest median income, with \$31,000, which is 25\% lower than the median of all degrees. While the unemployment rate is lower than the mean unemployment among all majors, the wages are likely to lead to a salary gap between men and women due to a disproportionate number of men and women willing to study these topics. The only degree in this cluster that has a higher percentage of male students is Theology and religious vocations, which is expected.

\subsection*{Conclusions}
The clustering strategy used separates clearly between STEM degrees and humanities in most clusters. If graduation income is a concern, students should pick degrees from clusters that have most majors from STEM. Similarly, governments should consider making these degrees more atractive for female students as the income disparity is driven by disproportionate percentages of male and female students in clusters of majors that have a higher than average income. Cluster 9 in Figure~\ref{fig:nine} offer the highest range of topics between majors while keeps the proportion of female to male students close to 1 and has median income close to the median of all majors. This represents the most balanced choice, that offers a bit of everything and students should consider if any of the majors in this cluster are interesting enough.

\newpage
\begin{thebibliography}{9}
    \bibitem{dailymail}
        dailymail.co.uk/news/article-10631871/Education-Secretary-Nadhim-Zahawi-plans-crackdown-Mickey-Mouse-degrees.html \emph{Education Secretary Nadhim Zahawi plans crackdown on 'Mickey Mouse' degrees - with universities required to publish drop-out rate and graduate job outcomes on every advert}
    \bibitem{acs}
        census.gov/programs-surveys/acs/microdata.html \emph{United States Census Bureau}
    \bibitem{fivethirtyeight}
        FiveThirtyEight.com \emph{American website that focuses on opinion poll analysis, politics, economics, and sports blogging}
    \bibitem{kmeanspca}
        https://ranger.uta.edu/~chqding/papers/KmeansPCA1.pdf
        \emph{K-means clustering via Principal Component Analysis - Chris Ding, Xiaofeng He}
    \bibitem{majorpick}
        fivethirtyeight.com/features/the-economic-guide-to-picking-a-college-major \emph{The Economic Guide To Picking A College Major}
    \bibitem{giniindex}
        \emph{An Introduction to Statistical Learning, Gini Index - James, Witten, Hastie, Tibshirani}

\end{thebibliography}

\newpage
\appendix
\section {Clusters}

\begin{center}
    \begin{tabular} {|| c c ||}
    \hline
    Major & Major category \\ [0.5ex]
    \hline\hline
    Metallurgical engineering & Engineering \\
    \hline
    Mining and mineral engineering & Engineering \\
    \hline
    Nuclear engineering & Engineering \\
    \hline
    Petroleum engineering & Engineering \\
    \hline
    \end {tabular}
    \captionof{table}{Cluster 1}
\end{center}

\begin{center}
    \begin{tabular} {|| c c ||}
    \hline
    Major & Major category \\ [0.5ex]
    \hline\hline

    Computer science & Computers \& mathematics \\
    \hline
    Mathematics & Computers \& mathematics \\
    \hline
    Civil engineering & Engineering \\
    \hline
    Electrical engineering & Engineering \\
    \hline
    General engineering & Engineering \\
    \hline
    Mechanical engineering & Engineering \\
    \hline
    \end {tabular}
    \captionof{table}{Cluster 2}
\end{center}

\begin{center}
    \begin{tabular} {|| c c ||}
    \hline
    Major & Major category \\ [0.5ex]
    \hline\hline
    Actuarial science & Business \\
    \hline
    Management information systems and statistics & Business \\
    \hline
    Operations logistics and e-commerce & Business \\
    \hline
    Computer and information systems & Computers \& mathematics \\
    \hline
    Information sciences & Computers \& mathematics \\
    \hline
    Mathematics and computer science & Computers \& mathematics \\
    \hline
    Aerospace engineering & Engineering \\
    \hline
    Architectural engineering & Engineering \\
    \hline
    Biological engineering & Engineering \\
    \hline
    Biomedical engineering & Engineering \\
    \hline
    Chemical engineering & Engineering \\
    \hline
    Computer engineering & Engineering \\
    \hline
    Electrical engineering technology & Engineering \\
    \hline
    Engineering mechanics physics and science & Engineering \\
    \hline
    Engineering technologies & Engineering \\
    \hline
    Environmental engineering & Engineering \\
    \hline
    Geological and geophysical engineering & Engineering \\
    \hline
    Industrial and manufacturing engineering & Engineering \\
    \hline
    Industrial production technologies & Engineering \\
    \hline
    Materials engineering and materials science & Engineering \\
    \hline
    Materials science & Engineering \\
    \hline
    Miscellaneous engineering & Engineering \\
    \hline
    Naval architecture and marine engineering & Engineering \\
    \hline
    Construction services & Industrial arts \& consumer services \\
    \hline
    Military technologies & Industrial arts \& consumer services \\
    \hline
    Court reporting & Law \& public policy \\
    \hline
    Public policy & Law \& public policy \\
    \hline
    Astronomy and astrophysics & Physical sciences \\
    \hline
    Physics & Physical sciences \\
    \hline
\end {tabular}
\captionof{table}{Cluster 3}
\label{table:three}
\end{center}

\begin{center}
    \begin{tabular} {|| c c ||}
    \hline
    Major & Major category \\ [0.5ex]
    \hline\hline
    Business management and administration & Business \\
    \hline
    Psychology & Psychology \& social work \\
    \hline
    \end {tabular}
    \captionof{table}{Cluster 4}
\end{center}

\begin{center}
    \begin{tabular} {|| c c ||}
    \hline
    Major & Major category \\ [0.5ex]
    \hline\hline
    Biology & Biology \& life science \\
    \hline
    General business & Business \\
    \hline
    Marketing and marketing research & Business \\
    \hline
    Communications & Communications \& journalism \\
    \hline
    Nursing & Health \\
    \hline
    English language and literature & Humanities \& liberal arts \\
    \hline
    \end {tabular}
    \captionof{table}{Cluster 5}
\end{center}

\begin{center}
    \begin{tabular} {|| c c ||}
    \hline
    Major & Major category \\ [0.5ex]
    \hline\hline

    Commercial art and graphic design & Arts \\
    \hline
    Accounting & Business \\
    \hline
    Finance & Business \\
    \hline
    Elementary education & Education \\
    \hline
    General education & Education \\
    \hline
    History & Humanities \& liberal arts \\
    \hline
    Physical fitness parks recreation and leisure & Industrial arts \& consumer services \\
    \hline
    Criminal justice and fire protection & Law \& public policy \\
    \hline
    Economics & Social science \\
    \hline
    Political science and government & Social science \\
    \hline
    Sociology & Social science \\
    \hline
    \end {tabular}
    \captionof{table}{Cluster 6}
\end{center}

\begin{center}
    \begin{tabular} {|| c c ||}
    \hline
    Major & Major category \\ [0.5ex]
    \hline\hline
    Drama and theater arts & Arts \\
    \hline
    Film video and photographic arts & Arts \\
    \hline
    Fine arts & Arts \\
    \hline
    Music & Arts \\
    \hline
    Hospitality management & Business \\
    \hline
    Advertising and public relations & Communications \& journalism \\
    \hline
    Journalism & Communications \& journalism \\
    \hline
    Mass media & Communications \& journalism \\
    \hline
    Architecture & Engineering \\
    \hline
    Treatment therapy professions & Health \\
    \hline
    Anthropology and archeology & Humanities \& liberal arts \\
    \hline
    Foreign language studies & Humanities \& liberal arts \\
    \hline
    Liberal arts & Humanities \& liberal arts \\
    \hline
    Philosophy and religious studies & Humanities \& liberal arts \\
    \hline
    Family and consumer sciences & Industrial arts \& consumer services \\
    \hline
    Chemistry & Physical sciences \\
    \hline
    Multi-disciplinary or general science & Physical sciences \\
    \hline
    Social work & Psychology \& social work \\
    \hline
    \end {tabular}
    \captionof{table}{Cluster 7}
\end{center}

\begin{center}
    \begin{longtable} {|| c c ||}
    \hline
    Major & Major category \\ [0.5ex]
    \hline\hline
    Animal sciences & Agriculture \& natural resources \\
    \hline
    Miscellaneous agriculture & Agriculture \& natural resources \\
    \hline
    Studio arts & Arts \\
    \hline
    Visual and performing arts & Arts \\
    \hline
    Ecology & Biology \& life science \\
    \hline
    Environmental science & Biology \& life science \\
    \hline
    Miscellaneous biology & Biology \& life science \\
    \hline
    Physiology & Biology \& life science \\
    \hline
    Zoology & Biology \& life science \\
    \hline
    Human resources and personnel management & Business \\
    \hline
    Art and music education & Education \\
    \hline
    Early childhood education & Education \\
    \hline
    Educational administration and supervision & Education \\
    \hline
    Language and drama education & Education \\
    \hline
    Library science & Education \\
    \hline
    Mathematics teacher education & Education \\
    \hline
    Physical and health education teaching & Education \\
    \hline
    Science and computer teacher education & Education \\
    \hline
    Secondary teacher education & Education \\
    \hline
    Social science or history teacher education & Education \\
    \hline
    Special needs education & Education \\
    \hline
    Teacher education: multiple levels & Education \\
    \hline
    Communication disorders sciences and services & Health \\
    \hline
    Community and public health & Health \\
    \hline
    General medical and health services & Health \\
    \hline
    Health and medical administrative services & Health \\
    \hline
    Health and medical preparatory programs & Health \\
    \hline
    Miscellaneous health medical professions & Health \\
    \hline
    Nutrition sciences & Health \\
    \hline
    Area ethnic and civilization studies & Humanities \& liberal arts \\
    \hline
    Art history and criticism & Humanities \& liberal arts \\
    \hline
    Composition and rhetoric & Humanities \& liberal arts \\
    \hline
    Humanities & Humanities \& liberal arts \\
    \hline
    Intercultural and international studies & Humanities \& liberal arts \\
    \hline
    Linguistics and comparative language and literature & Humanities \& liberal arts \\
    \hline
    Other foreign languages & Humanities \& liberal arts \\
    \hline
    Theology and religious vocations & Humanities \& liberal arts \\
    \hline
    Cosmetology services and culinary arts & Industrial arts \& consumer services \\
    \hline
    Multi/interdisciplinary studies & Interdisciplinary \\
    \hline
    Geosciences & Physical sciences \\
    \hline
    Clinical psychology & Psychology \& social work \\
    \hline
    Counseling psychology & Psychology \& social work \\
    \hline
    Educational psychology & Psychology \& social work \\
    \hline
    Human services and community organization & Psychology \& social work \\
    \hline
    Miscellaneous psychology & Psychology \& social work \\
    \hline
    Criminology & Social science \\
    \hline
    General social sciences & Social science \\
    \hline
    Interdisciplinary social sciences & Social science \\
    \hline
\end {longtable}
\captionof{table}{Cluster 8}
\label{table:eight}
\end{center}

\begin{center}
    \begin{longtable} {|| c c ||}
    \hline
    Major & Major category \\ [0.5ex]
    \hline\hline

    Agricultural economics & Agriculture \& natural resources \\
    \hline
    Agriculture production and management & Agriculture \& natural resources \\
    \hline
    Forestry & Agriculture \& natural resources \\
    \hline
    General agriculture & Agriculture \& natural resources \\
    \hline
    Natural resources management & Agriculture \& natural resources \\
    \hline
    Plant science and agronomy & Agriculture \& natural resources \\
    \hline
    Soil science & Agriculture \& natural resources \\
    \hline
    Miscellaneous fine arts & Arts \\
    \hline
    Biochemical sciences & Biology \& life science \\
    \hline
    Botany & Biology \& life science \\
    \hline
    Cognitive science and biopsychology & Biology \& life science \\
    \hline
    Genetics & Biology \& life science \\
    \hline
    Microbiology & Biology \& life science \\
    \hline
    Molecular biology & Biology \& life science \\
    \hline
    Neuroscience & Biology \& life science \\
    \hline
    Pharmacology & Biology \& life science \\
    \hline
    Business economics & Business \\
    \hline
    International business & Business \\
    \hline
    Miscellaneous business \& medical administration & Business \\
    \hline
    Applied mathematics & Computers \& mathematics \\
    \hline
    Communication technologies & Computers \& mathematics \\
    \hline
    Computer administration management and security & Computers \& mathematics \\
    \hline
    Computer networking and telecommunications & Computers \& mathematics \\
    \hline
    Computer programming and data processing & Computers \& mathematics \\
    \hline
    Statistics and decision science & Computers \& mathematics \\
    \hline
    Miscellaneous education & Education \\
    \hline
    School student counseling & Education \\
    \hline
    Engineering and industrial management & Engineering \\
    \hline
    Mechanical engineering related technologies & Engineering \\
    \hline
    Miscellaneous engineering technologies & Engineering \\
    \hline
    Medical assisting services & Health \\
    \hline
    Medical technologies technicians & Health \\
    \hline
    Pharmacy pharmaceutical sciences and administration & Health \\
    \hline
    United states history & Humanities \& liberal arts \\
    \hline
    Electrical, mechanical, and precision technologies & Industrial arts \& consumer services \\
    \hline
    Transportation sciences and technologies & Industrial arts \& consumer services \\
    \hline
    Pre-law and legal studies & Law \& public policy \\
    \hline
    Public administration & Law \& public policy \\
    \hline
    Atmospheric sciences and meteorology & Physical sciences \\
    \hline
    Geology and earth science & Physical sciences \\
    \hline
    Nuclear, industrial radiology, and biological technologies & Physical sciences \\
    \hline
    Oceanography & Physical sciences \\
    \hline
    Physical sciences & Physical sciences \\
    \hline
    Industrial and organizational psychology & Psychology \& social work \\
    \hline
    Social psychology & Psychology \& social work \\
    \hline
    Geography & Social science \\
    \hline
    International relations & Social science \\
    \hline
    Miscellaneous social sciences & Social science \\
    \hline
\end {longtable}
\captionof{table}{Cluster 9}
\label{fig:nine}
\end{center}
\end{document}\models