\documentclass[11pt]{article}

    \usepackage[breakable]{tcolorbox}
    \usepackage{parskip} % Stop auto-indenting (to mimic markdown behaviour)
    
    \usepackage{iftex}
    \ifPDFTeX
    	\usepackage[T1]{fontenc}
    	\usepackage{mathpazo}
    \else
    	\usepackage{fontspec}
    \fi

    % Basic figure setup, for now with no caption control since it's done
    % automatically by Pandoc (which extracts ![](path) syntax from Markdown).
    \usepackage{graphicx}
    % Maintain compatibility with old templates. Remove in nbconvert 6.0
    \let\Oldincludegraphics\includegraphics
    % Ensure that by default, figures have no caption (until we provide a
    % proper Figure object with a Caption API and a way to capture that
    % in the conversion process - todo).
    \usepackage{caption}
    \DeclareCaptionFormat{nocaption}{}
    \captionsetup{format=nocaption,aboveskip=0pt,belowskip=0pt}

    \usepackage{float}
    \floatplacement{figure}{H} % forces figures to be placed at the correct location
    \usepackage{xcolor} % Allow colors to be defined
    \usepackage{enumerate} % Needed for markdown enumerations to work
    \usepackage{geometry} % Used to adjust the document margins
    \usepackage{amsmath} % Equations
    \usepackage{amssymb} % Equations
    \usepackage{textcomp} % defines textquotesingle
    % Hack from http://tex.stackexchange.com/a/47451/13684:
    \AtBeginDocument{%
        \def\PYZsq{\textquotesingle}% Upright quotes in Pygmentized code
    }
    \usepackage{upquote} % Upright quotes for verbatim code
    \usepackage{eurosym} % defines \euro
    \usepackage[mathletters]{ucs} % Extended unicode (utf-8) support
    \usepackage{fancyvrb} % verbatim replacement that allows latex
    \usepackage{grffile} % extends the file name processing of package graphics 
                         % to support a larger range
    \makeatletter % fix for old versions of grffile with XeLaTeX
    \@ifpackagelater{grffile}{2019/11/01}
    {
      % Do nothing on new versions
    }
    {
      \def\Gread@@xetex#1{%
        \IfFileExists{"\Gin@base".bb}%
        {\Gread@eps{\Gin@base.bb}}%
        {\Gread@@xetex@aux#1}%
      }
    }
    \makeatother
    \usepackage[Export]{adjustbox} % Used to constrain images to a maximum size
    \adjustboxset{max size={0.9\linewidth}{0.9\paperheight}}

    % The hyperref package gives us a pdf with properly built
    % internal navigation ('pdf bookmarks' for the table of contents,
    % internal cross-reference links, web links for URLs, etc.)
    \usepackage{hyperref}
    % The default LaTeX title has an obnoxious amount of whitespace. By default,
    % titling removes some of it. It also provides customization options.
    \usepackage{titling}
    \usepackage{longtable} % longtable support required by pandoc >1.10
    \usepackage{booktabs}  % table support for pandoc > 1.12.2
    \usepackage[inline]{enumitem} % IRkernel/repr support (it uses the enumerate* environment)
    \usepackage[normalem]{ulem} % ulem is needed to support strikethroughs (\sout)
                                % normalem makes italics be italics, not underlines
    \usepackage{mathrsfs}
    

    
    % Colors for the hyperref package
    \definecolor{urlcolor}{rgb}{0,.145,.698}
    \definecolor{linkcolor}{rgb}{.71,0.21,0.01}
    \definecolor{citecolor}{rgb}{.12,.54,.11}

    % ANSI colors
    \definecolor{ansi-black}{HTML}{3E424D}
    \definecolor{ansi-black-intense}{HTML}{282C36}
    \definecolor{ansi-red}{HTML}{E75C58}
    \definecolor{ansi-red-intense}{HTML}{B22B31}
    \definecolor{ansi-green}{HTML}{00A250}
    \definecolor{ansi-green-intense}{HTML}{007427}
    \definecolor{ansi-yellow}{HTML}{DDB62B}
    \definecolor{ansi-yellow-intense}{HTML}{B27D12}
    \definecolor{ansi-blue}{HTML}{208FFB}
    \definecolor{ansi-blue-intense}{HTML}{0065CA}
    \definecolor{ansi-magenta}{HTML}{D160C4}
    \definecolor{ansi-magenta-intense}{HTML}{A03196}
    \definecolor{ansi-cyan}{HTML}{60C6C8}
    \definecolor{ansi-cyan-intense}{HTML}{258F8F}
    \definecolor{ansi-white}{HTML}{C5C1B4}
    \definecolor{ansi-white-intense}{HTML}{A1A6B2}
    \definecolor{ansi-default-inverse-fg}{HTML}{FFFFFF}
    \definecolor{ansi-default-inverse-bg}{HTML}{000000}

    % common color for the border for error outputs.
    \definecolor{outerrorbackground}{HTML}{FFDFDF}

    % commands and environments needed by pandoc snippets
    % extracted from the output of `pandoc -s`
    \providecommand{\tightlist}{%
      \setlength{\itemsep}{0pt}\setlength{\parskip}{0pt}}
    \DefineVerbatimEnvironment{Highlighting}{Verbatim}{commandchars=\\\{\}}
    % Add ',fontsize=\small' for more characters per line
    \newenvironment{Shaded}{}{}
    \newcommand{\KeywordTok}[1]{\textcolor[rgb]{0.00,0.44,0.13}{\textbf{{#1}}}}
    \newcommand{\DataTypeTok}[1]{\textcolor[rgb]{0.56,0.13,0.00}{{#1}}}
    \newcommand{\DecValTok}[1]{\textcolor[rgb]{0.25,0.63,0.44}{{#1}}}
    \newcommand{\BaseNTok}[1]{\textcolor[rgb]{0.25,0.63,0.44}{{#1}}}
    \newcommand{\FloatTok}[1]{\textcolor[rgb]{0.25,0.63,0.44}{{#1}}}
    \newcommand{\CharTok}[1]{\textcolor[rgb]{0.25,0.44,0.63}{{#1}}}
    \newcommand{\StringTok}[1]{\textcolor[rgb]{0.25,0.44,0.63}{{#1}}}
    \newcommand{\CommentTok}[1]{\textcolor[rgb]{0.38,0.63,0.69}{\textit{{#1}}}}
    \newcommand{\OtherTok}[1]{\textcolor[rgb]{0.00,0.44,0.13}{{#1}}}
    \newcommand{\AlertTok}[1]{\textcolor[rgb]{1.00,0.00,0.00}{\textbf{{#1}}}}
    \newcommand{\FunctionTok}[1]{\textcolor[rgb]{0.02,0.16,0.49}{{#1}}}
    \newcommand{\RegionMarkerTok}[1]{{#1}}
    \newcommand{\ErrorTok}[1]{\textcolor[rgb]{1.00,0.00,0.00}{\textbf{{#1}}}}
    \newcommand{\NormalTok}[1]{{#1}}
    
    % Additional commands for more recent versions of Pandoc
    \newcommand{\ConstantTok}[1]{\textcolor[rgb]{0.53,0.00,0.00}{{#1}}}
    \newcommand{\SpecialCharTok}[1]{\textcolor[rgb]{0.25,0.44,0.63}{{#1}}}
    \newcommand{\VerbatimStringTok}[1]{\textcolor[rgb]{0.25,0.44,0.63}{{#1}}}
    \newcommand{\SpecialStringTok}[1]{\textcolor[rgb]{0.73,0.40,0.53}{{#1}}}
    \newcommand{\ImportTok}[1]{{#1}}
    \newcommand{\DocumentationTok}[1]{\textcolor[rgb]{0.73,0.13,0.13}{\textit{{#1}}}}
    \newcommand{\AnnotationTok}[1]{\textcolor[rgb]{0.38,0.63,0.69}{\textbf{\textit{{#1}}}}}
    \newcommand{\CommentVarTok}[1]{\textcolor[rgb]{0.38,0.63,0.69}{\textbf{\textit{{#1}}}}}
    \newcommand{\VariableTok}[1]{\textcolor[rgb]{0.10,0.09,0.49}{{#1}}}
    \newcommand{\ControlFlowTok}[1]{\textcolor[rgb]{0.00,0.44,0.13}{\textbf{{#1}}}}
    \newcommand{\OperatorTok}[1]{\textcolor[rgb]{0.40,0.40,0.40}{{#1}}}
    \newcommand{\BuiltInTok}[1]{{#1}}
    \newcommand{\ExtensionTok}[1]{{#1}}
    \newcommand{\PreprocessorTok}[1]{\textcolor[rgb]{0.74,0.48,0.00}{{#1}}}
    \newcommand{\AttributeTok}[1]{\textcolor[rgb]{0.49,0.56,0.16}{{#1}}}
    \newcommand{\InformationTok}[1]{\textcolor[rgb]{0.38,0.63,0.69}{\textbf{\textit{{#1}}}}}
    \newcommand{\WarningTok}[1]{\textcolor[rgb]{0.38,0.63,0.69}{\textbf{\textit{{#1}}}}}
    
    
    % Define a nice break command that doesn't care if a line doesn't already
    % exist.
    \def\br{\hspace*{\fill} \\* }
    % Math Jax compatibility definitions
    \def\gt{>}
    \def\lt{<}
    \let\Oldtex\TeX
    \let\Oldlatex\LaTeX
    \renewcommand{\TeX}{\textrm{\Oldtex}}
    \renewcommand{\LaTeX}{\textrm{\Oldlatex}}
    % Document parameters
    % Document title
    \title{}
    \date{}
    
    
    
% Pygments definitions
\makeatletter
\def\PY@reset{\let\PY@it=\relax \let\PY@bf=\relax%
    \let\PY@ul=\relax \let\PY@tc=\relax%
    \let\PY@bc=\relax \let\PY@ff=\relax}
\def\PY@tok#1{\csname PY@tok@#1\endcsname}
\def\PY@toks#1+{\ifx\relax#1\empty\else%
    \PY@tok{#1}\expandafter\PY@toks\fi}
\def\PY@do#1{\PY@bc{\PY@tc{\PY@ul{%
    \PY@it{\PY@bf{\PY@ff{#1}}}}}}}
\def\PY#1#2{\PY@reset\PY@toks#1+\relax+\PY@do{#2}}

\@namedef{PY@tok@w}{\def\PY@tc##1{\textcolor[rgb]{0.73,0.73,0.73}{##1}}}
\@namedef{PY@tok@c}{\let\PY@it=\textit\def\PY@tc##1{\textcolor[rgb]{0.25,0.50,0.50}{##1}}}
\@namedef{PY@tok@cp}{\def\PY@tc##1{\textcolor[rgb]{0.74,0.48,0.00}{##1}}}
\@namedef{PY@tok@k}{\let\PY@bf=\textbf\def\PY@tc##1{\textcolor[rgb]{0.00,0.50,0.00}{##1}}}
\@namedef{PY@tok@kp}{\def\PY@tc##1{\textcolor[rgb]{0.00,0.50,0.00}{##1}}}
\@namedef{PY@tok@kt}{\def\PY@tc##1{\textcolor[rgb]{0.69,0.00,0.25}{##1}}}
\@namedef{PY@tok@o}{\def\PY@tc##1{\textcolor[rgb]{0.40,0.40,0.40}{##1}}}
\@namedef{PY@tok@ow}{\let\PY@bf=\textbf\def\PY@tc##1{\textcolor[rgb]{0.67,0.13,1.00}{##1}}}
\@namedef{PY@tok@nb}{\def\PY@tc##1{\textcolor[rgb]{0.00,0.50,0.00}{##1}}}
\@namedef{PY@tok@nf}{\def\PY@tc##1{\textcolor[rgb]{0.00,0.00,1.00}{##1}}}
\@namedef{PY@tok@nc}{\let\PY@bf=\textbf\def\PY@tc##1{\textcolor[rgb]{0.00,0.00,1.00}{##1}}}
\@namedef{PY@tok@nn}{\let\PY@bf=\textbf\def\PY@tc##1{\textcolor[rgb]{0.00,0.00,1.00}{##1}}}
\@namedef{PY@tok@ne}{\let\PY@bf=\textbf\def\PY@tc##1{\textcolor[rgb]{0.82,0.25,0.23}{##1}}}
\@namedef{PY@tok@nv}{\def\PY@tc##1{\textcolor[rgb]{0.10,0.09,0.49}{##1}}}
\@namedef{PY@tok@no}{\def\PY@tc##1{\textcolor[rgb]{0.53,0.00,0.00}{##1}}}
\@namedef{PY@tok@nl}{\def\PY@tc##1{\textcolor[rgb]{0.63,0.63,0.00}{##1}}}
\@namedef{PY@tok@ni}{\let\PY@bf=\textbf\def\PY@tc##1{\textcolor[rgb]{0.60,0.60,0.60}{##1}}}
\@namedef{PY@tok@na}{\def\PY@tc##1{\textcolor[rgb]{0.49,0.56,0.16}{##1}}}
\@namedef{PY@tok@nt}{\let\PY@bf=\textbf\def\PY@tc##1{\textcolor[rgb]{0.00,0.50,0.00}{##1}}}
\@namedef{PY@tok@nd}{\def\PY@tc##1{\textcolor[rgb]{0.67,0.13,1.00}{##1}}}
\@namedef{PY@tok@s}{\def\PY@tc##1{\textcolor[rgb]{0.73,0.13,0.13}{##1}}}
\@namedef{PY@tok@sd}{\let\PY@it=\textit\def\PY@tc##1{\textcolor[rgb]{0.73,0.13,0.13}{##1}}}
\@namedef{PY@tok@si}{\let\PY@bf=\textbf\def\PY@tc##1{\textcolor[rgb]{0.73,0.40,0.53}{##1}}}
\@namedef{PY@tok@se}{\let\PY@bf=\textbf\def\PY@tc##1{\textcolor[rgb]{0.73,0.40,0.13}{##1}}}
\@namedef{PY@tok@sr}{\def\PY@tc##1{\textcolor[rgb]{0.73,0.40,0.53}{##1}}}
\@namedef{PY@tok@ss}{\def\PY@tc##1{\textcolor[rgb]{0.10,0.09,0.49}{##1}}}
\@namedef{PY@tok@sx}{\def\PY@tc##1{\textcolor[rgb]{0.00,0.50,0.00}{##1}}}
\@namedef{PY@tok@m}{\def\PY@tc##1{\textcolor[rgb]{0.40,0.40,0.40}{##1}}}
\@namedef{PY@tok@gh}{\let\PY@bf=\textbf\def\PY@tc##1{\textcolor[rgb]{0.00,0.00,0.50}{##1}}}
\@namedef{PY@tok@gu}{\let\PY@bf=\textbf\def\PY@tc##1{\textcolor[rgb]{0.50,0.00,0.50}{##1}}}
\@namedef{PY@tok@gd}{\def\PY@tc##1{\textcolor[rgb]{0.63,0.00,0.00}{##1}}}
\@namedef{PY@tok@gi}{\def\PY@tc##1{\textcolor[rgb]{0.00,0.63,0.00}{##1}}}
\@namedef{PY@tok@gr}{\def\PY@tc##1{\textcolor[rgb]{1.00,0.00,0.00}{##1}}}
\@namedef{PY@tok@ge}{\let\PY@it=\textit}
\@namedef{PY@tok@gs}{\let\PY@bf=\textbf}
\@namedef{PY@tok@gp}{\let\PY@bf=\textbf\def\PY@tc##1{\textcolor[rgb]{0.00,0.00,0.50}{##1}}}
\@namedef{PY@tok@go}{\def\PY@tc##1{\textcolor[rgb]{0.53,0.53,0.53}{##1}}}
\@namedef{PY@tok@gt}{\def\PY@tc##1{\textcolor[rgb]{0.00,0.27,0.87}{##1}}}
\@namedef{PY@tok@err}{\def\PY@bc##1{{\setlength{\fboxsep}{\string -\fboxrule}\fcolorbox[rgb]{1.00,0.00,0.00}{1,1,1}{\strut ##1}}}}
\@namedef{PY@tok@kc}{\let\PY@bf=\textbf\def\PY@tc##1{\textcolor[rgb]{0.00,0.50,0.00}{##1}}}
\@namedef{PY@tok@kd}{\let\PY@bf=\textbf\def\PY@tc##1{\textcolor[rgb]{0.00,0.50,0.00}{##1}}}
\@namedef{PY@tok@kn}{\let\PY@bf=\textbf\def\PY@tc##1{\textcolor[rgb]{0.00,0.50,0.00}{##1}}}
\@namedef{PY@tok@kr}{\let\PY@bf=\textbf\def\PY@tc##1{\textcolor[rgb]{0.00,0.50,0.00}{##1}}}
\@namedef{PY@tok@bp}{\def\PY@tc##1{\textcolor[rgb]{0.00,0.50,0.00}{##1}}}
\@namedef{PY@tok@fm}{\def\PY@tc##1{\textcolor[rgb]{0.00,0.00,1.00}{##1}}}
\@namedef{PY@tok@vc}{\def\PY@tc##1{\textcolor[rgb]{0.10,0.09,0.49}{##1}}}
\@namedef{PY@tok@vg}{\def\PY@tc##1{\textcolor[rgb]{0.10,0.09,0.49}{##1}}}
\@namedef{PY@tok@vi}{\def\PY@tc##1{\textcolor[rgb]{0.10,0.09,0.49}{##1}}}
\@namedef{PY@tok@vm}{\def\PY@tc##1{\textcolor[rgb]{0.10,0.09,0.49}{##1}}}
\@namedef{PY@tok@sa}{\def\PY@tc##1{\textcolor[rgb]{0.73,0.13,0.13}{##1}}}
\@namedef{PY@tok@sb}{\def\PY@tc##1{\textcolor[rgb]{0.73,0.13,0.13}{##1}}}
\@namedef{PY@tok@sc}{\def\PY@tc##1{\textcolor[rgb]{0.73,0.13,0.13}{##1}}}
\@namedef{PY@tok@dl}{\def\PY@tc##1{\textcolor[rgb]{0.73,0.13,0.13}{##1}}}
\@namedef{PY@tok@s2}{\def\PY@tc##1{\textcolor[rgb]{0.73,0.13,0.13}{##1}}}
\@namedef{PY@tok@sh}{\def\PY@tc##1{\textcolor[rgb]{0.73,0.13,0.13}{##1}}}
\@namedef{PY@tok@s1}{\def\PY@tc##1{\textcolor[rgb]{0.73,0.13,0.13}{##1}}}
\@namedef{PY@tok@mb}{\def\PY@tc##1{\textcolor[rgb]{0.40,0.40,0.40}{##1}}}
\@namedef{PY@tok@mf}{\def\PY@tc##1{\textcolor[rgb]{0.40,0.40,0.40}{##1}}}
\@namedef{PY@tok@mh}{\def\PY@tc##1{\textcolor[rgb]{0.40,0.40,0.40}{##1}}}
\@namedef{PY@tok@mi}{\def\PY@tc##1{\textcolor[rgb]{0.40,0.40,0.40}{##1}}}
\@namedef{PY@tok@il}{\def\PY@tc##1{\textcolor[rgb]{0.40,0.40,0.40}{##1}}}
\@namedef{PY@tok@mo}{\def\PY@tc##1{\textcolor[rgb]{0.40,0.40,0.40}{##1}}}
\@namedef{PY@tok@ch}{\let\PY@it=\textit\def\PY@tc##1{\textcolor[rgb]{0.25,0.50,0.50}{##1}}}
\@namedef{PY@tok@cm}{\let\PY@it=\textit\def\PY@tc##1{\textcolor[rgb]{0.25,0.50,0.50}{##1}}}
\@namedef{PY@tok@cpf}{\let\PY@it=\textit\def\PY@tc##1{\textcolor[rgb]{0.25,0.50,0.50}{##1}}}
\@namedef{PY@tok@c1}{\let\PY@it=\textit\def\PY@tc##1{\textcolor[rgb]{0.25,0.50,0.50}{##1}}}
\@namedef{PY@tok@cs}{\let\PY@it=\textit\def\PY@tc##1{\textcolor[rgb]{0.25,0.50,0.50}{##1}}}

\def\PYZbs{\char`\\}
\def\PYZus{\char`\_}
\def\PYZob{\char`\{}
\def\PYZcb{\char`\}}
\def\PYZca{\char`\^}
\def\PYZam{\char`\&}
\def\PYZlt{\char`\<}
\def\PYZgt{\char`\>}
\def\PYZsh{\char`\#}
\def\PYZpc{\char`\%}
\def\PYZdl{\char`\$}
\def\PYZhy{\char`\-}
\def\PYZsq{\char`\'}
\def\PYZdq{\char`\"}
\def\PYZti{\char`\~}
% for compatibility with earlier versions
\def\PYZat{@}
\def\PYZlb{[}
\def\PYZrb{]}
\makeatother


    % For linebreaks inside Verbatim environment from package fancyvrb. 
    \makeatletter
        \newbox\Wrappedcontinuationbox 
        \newbox\Wrappedvisiblespacebox 
        \newcommand*\Wrappedvisiblespace {\textcolor{red}{\textvisiblespace}} 
        \newcommand*\Wrappedcontinuationsymbol {\textcolor{red}{\llap{\tiny$\m@th\hookrightarrow$}}} 
        \newcommand*\Wrappedcontinuationindent {3ex } 
        \newcommand*\Wrappedafterbreak {\kern\Wrappedcontinuationindent\copy\Wrappedcontinuationbox} 
        % Take advantage of the already applied Pygments mark-up to insert 
        % potential linebreaks for TeX processing. 
        %        {, <, #, %, $, ' and ": go to next line. 
        %        _, }, ^, &, >, - and ~: stay at end of broken line. 
        % Use of \textquotesingle for straight quote. 
        \newcommand*\Wrappedbreaksatspecials {% 
            \def\PYGZus{\discretionary{\char`\_}{\Wrappedafterbreak}{\char`\_}}% 
            \def\PYGZob{\discretionary{}{\Wrappedafterbreak\char`\{}{\char`\{}}% 
            \def\PYGZcb{\discretionary{\char`\}}{\Wrappedafterbreak}{\char`\}}}% 
            \def\PYGZca{\discretionary{\char`\^}{\Wrappedafterbreak}{\char`\^}}% 
            \def\PYGZam{\discretionary{\char`\&}{\Wrappedafterbreak}{\char`\&}}% 
            \def\PYGZlt{\discretionary{}{\Wrappedafterbreak\char`\<}{\char`\<}}% 
            \def\PYGZgt{\discretionary{\char`\>}{\Wrappedafterbreak}{\char`\>}}% 
            \def\PYGZsh{\discretionary{}{\Wrappedafterbreak\char`\#}{\char`\#}}% 
            \def\PYGZpc{\discretionary{}{\Wrappedafterbreak\char`\%}{\char`\%}}% 
            \def\PYGZdl{\discretionary{}{\Wrappedafterbreak\char`\$}{\char`\$}}% 
            \def\PYGZhy{\discretionary{\char`\-}{\Wrappedafterbreak}{\char`\-}}% 
            \def\PYGZsq{\discretionary{}{\Wrappedafterbreak\textquotesingle}{\textquotesingle}}% 
            \def\PYGZdq{\discretionary{}{\Wrappedafterbreak\char`\"}{\char`\"}}% 
            \def\PYGZti{\discretionary{\char`\~}{\Wrappedafterbreak}{\char`\~}}% 
        } 
        % Some characters . , ; ? ! / are not pygmentized. 
        % This macro makes them "active" and they will insert potential linebreaks 
        \newcommand*\Wrappedbreaksatpunct {% 
            \lccode`\~`\.\lowercase{\def~}{\discretionary{\hbox{\char`\.}}{\Wrappedafterbreak}{\hbox{\char`\.}}}% 
            \lccode`\~`\,\lowercase{\def~}{\discretionary{\hbox{\char`\,}}{\Wrappedafterbreak}{\hbox{\char`\,}}}% 
            \lccode`\~`\;\lowercase{\def~}{\discretionary{\hbox{\char`\;}}{\Wrappedafterbreak}{\hbox{\char`\;}}}% 
            \lccode`\~`\:\lowercase{\def~}{\discretionary{\hbox{\char`\:}}{\Wrappedafterbreak}{\hbox{\char`\:}}}% 
            \lccode`\~`\?\lowercase{\def~}{\discretionary{\hbox{\char`\?}}{\Wrappedafterbreak}{\hbox{\char`\?}}}% 
            \lccode`\~`\!\lowercase{\def~}{\discretionary{\hbox{\char`\!}}{\Wrappedafterbreak}{\hbox{\char`\!}}}% 
            \lccode`\~`\/\lowercase{\def~}{\discretionary{\hbox{\char`\/}}{\Wrappedafterbreak}{\hbox{\char`\/}}}% 
            \catcode`\.\active
            \catcode`\,\active 
            \catcode`\;\active
            \catcode`\:\active
            \catcode`\?\active
            \catcode`\!\active
            \catcode`\/\active 
            \lccode`\~`\~ 	
        }
    \makeatother

    \let\OriginalVerbatim=\Verbatim
    \makeatletter
    \renewcommand{\Verbatim}[1][1]{%
        %\parskip\z@skip
        \sbox\Wrappedcontinuationbox {\Wrappedcontinuationsymbol}%
        \sbox\Wrappedvisiblespacebox {\FV@SetupFont\Wrappedvisiblespace}%
        \def\FancyVerbFormatLine ##1{\hsize\linewidth
            \vtop{\raggedright\hyphenpenalty\z@\exhyphenpenalty\z@
                \doublehyphendemerits\z@\finalhyphendemerits\z@
                \strut ##1\strut}%
        }%
        % If the linebreak is at a space, the latter will be displayed as visible
        % space at end of first line, and a continuation symbol starts next line.
        % Stretch/shrink are however usually zero for typewriter font.
        \def\FV@Space {%
            \nobreak\hskip\z@ plus\fontdimen3\font minus\fontdimen4\font
            \discretionary{\copy\Wrappedvisiblespacebox}{\Wrappedafterbreak}
            {\kern\fontdimen2\font}%
        }%
        
        % Allow breaks at special characters using \PYG... macros.
        \Wrappedbreaksatspecials
        % Breaks at punctuation characters . , ; ? ! and / need catcode=\active 	
        \OriginalVerbatim[#1,codes*=\Wrappedbreaksatpunct]%
    }
    \makeatother

    % Exact colors from NB
    \definecolor{incolor}{HTML}{303F9F}
    \definecolor{outcolor}{HTML}{D84315}
    \definecolor{cellborder}{HTML}{CFCFCF}
    \definecolor{cellbackground}{HTML}{F7F7F7}
    
    % prompt
    \makeatletter
    \newcommand{\boxspacing}{\kern\kvtcb@left@rule\kern\kvtcb@boxsep}
    \makeatother
    \newcommand{\prompt}[4]{
        {\ttfamily\llap{{\color{#2}[#3]:\hspace{3pt}#4}}\vspace{-\baselineskip}}
    }
    

    
    % Prevent overflowing lines due to hard-to-break entities
    \sloppy 
    % Setup hyperref package
    \hypersetup{
      breaklinks=true,  % so long urls are correctly broken across lines
      colorlinks=true,
      urlcolor=urlcolor,
      linkcolor=linkcolor,
      citecolor=citecolor,
      }
    % Slightly bigger margins than the latex defaults
    
    \geometry{verbose,tmargin=0in,bmargin=1in,lmargin=1in,rmargin=1in}
    
    

\begin{document}
    
    \maketitle
    
    

    
    \begin{tcolorbox}[breakable, size=fbox, boxrule=1pt, pad at break*=1mm,colback=cellbackground, colframe=cellborder]
\prompt{In}{incolor}{ }{\boxspacing}
\begin{Verbatim}[commandchars=\\\{\}]
\PY{n+nf}{require}\PY{p}{(}\PY{n}{ggplot}\PY{p}{)}
\PY{n+nf}{require}\PY{p}{(}\PY{n}{tidyverse}\PY{p}{)}
\PY{n+nf}{require}\PY{p}{(}\PY{n}{ggcorrplot}\PY{p}{)}
\PY{n+nf}{require}\PY{p}{(}\PY{n}{ggthemes}\PY{p}{)}
\PY{n+nf}{require}\PY{p}{(}\PY{n}{cluster}\PY{p}{)}
\PY{n+nf}{require}\PY{p}{(}\PY{n}{corrplot}\PY{p}{)}
\PY{n+nf}{require}\PY{p}{(}\PY{n}{factoextra}\PY{p}{)}
\PY{n+nf}{require}\PY{p}{(}\PY{n}{cowplot}\PY{p}{)}
\end{Verbatim}
\end{tcolorbox}

    \begin{tcolorbox}[breakable, size=fbox, boxrule=1pt, pad at break*=1mm,colback=cellbackground, colframe=cellborder]
\prompt{In}{incolor}{2}{\boxspacing}
\begin{Verbatim}[commandchars=\\\{\}]
\PY{n}{palette} \PY{o}{=} \PY{n+nf}{c}\PY{p}{(}
    \PY{l+s}{\PYZdq{}}\PY{l+s}{\PYZsh{}2E9FDF\PYZdq{}}\PY{p}{,} \PY{l+s}{\PYZdq{}}\PY{l+s}{\PYZsh{}4782b3\PYZdq{}}\PY{p}{,} \PY{l+s}{\PYZdq{}}\PY{l+s}{\PYZsh{}E7B800\PYZdq{}}\PY{p}{,}
    \PY{l+s}{\PYZdq{}}\PY{l+s}{\PYZsh{}66acff\PYZdq{}}\PY{p}{,} \PY{l+s}{\PYZdq{}}\PY{l+s}{\PYZsh{}fff566\PYZdq{}}\PY{p}{,} \PY{l+s}{\PYZdq{}}\PY{l+s}{\PYZsh{}b34766\PYZdq{}}\PY{p}{,}
    \PY{l+s}{\PYZdq{}}\PY{l+s}{\PYZsh{}7a327d\PYZdq{}}\PY{p}{,} \PY{l+s}{\PYZdq{}}\PY{l+s}{\PYZsh{}66acff\PYZdq{}}\PY{p}{,} \PY{l+s}{\PYZdq{}}\PY{l+s}{\PYZsh{}ff6692\PYZdq{}}\PY{p}{,}
    \PY{l+s}{\PYZdq{}}\PY{l+s}{\PYZsh{}b3ab47\PYZdq{}}\PY{p}{,} \PY{l+s}{\PYZdq{}}\PY{l+s}{\PYZsh{}ffb3d7\PYZdq{}}\PY{p}{,} \PY{l+s}{\PYZdq{}}\PY{l+s}{\PYZsh{}66faff\PYZdq{}}\PY{p}{,}
    \PY{l+s}{\PYZdq{}}\PY{l+s}{\PYZsh{}7d7632\PYZdq{}}\PY{p}{,} \PY{l+s}{\PYZdq{}}\PY{l+s}{\PYZsh{}00AFBB\PYZdq{}}\PY{p}{,} \PY{l+s}{\PYZdq{}}\PY{l+s}{\PYZsh{}002db3\PYZdq{}}\PY{p}{,}
    \PY{l+s}{\PYZdq{}}\PY{l+s}{\PYZsh{}ff0000\PYZdq{}}
\PY{p}{)}
\end{Verbatim}
\end{tcolorbox}

    \hypertarget{drop-categorical-and-index-column-from-pca}{%
\subsubsection{Drop categorical and index column from
PCA}\label{drop-categorical-and-index-column-from-pca}}

    \begin{tcolorbox}[breakable, size=fbox, boxrule=1pt, pad at break*=1mm,colback=cellbackground, colframe=cellborder]
\prompt{In}{incolor}{3}{\boxspacing}
\begin{Verbatim}[commandchars=\\\{\}]
\PY{n}{data} \PY{o}{=} \PY{n+nf}{read.csv}\PY{p}{(}\PY{l+s}{\PYZdq{}}\PY{l+s}{college.csv\PYZdq{}}\PY{p}{,} \PY{n}{header}\PY{o}{=}\PY{k+kc}{TRUE}\PY{p}{,} \PY{n}{sep}\PY{o}{=}\PY{l+s}{\PYZdq{}}\PY{l+s}{,\PYZdq{}}\PY{p}{)}
\end{Verbatim}
\end{tcolorbox}

    \begin{tcolorbox}[breakable, size=fbox, boxrule=1pt, pad at break*=1mm,colback=cellbackground, colframe=cellborder]
\prompt{In}{incolor}{4}{\boxspacing}
\begin{Verbatim}[commandchars=\\\{\}]
\PY{n}{data} \PY{o}{=} \PY{n}{data} \PY{o}{\PYZpc{}\PYZgt{}\PYZpc{}} \PY{n+nf}{drop\PYZus{}na}\PY{p}{(}\PY{p}{)}
\PY{n}{subdata} \PY{o}{=} \PY{n}{data}\PY{p}{[}\PY{o}{\PYZhy{}}\PY{n+nf}{c}\PY{p}{(}\PY{l+m}{1}\PY{p}{,} \PY{l+m}{2}\PY{p}{,} \PY{l+m}{3}\PY{p}{,} \PY{l+m}{7}\PY{p}{)}\PY{p}{]}
\PY{n}{subdata} \PY{o}{=} \PY{n}{subdata}
\end{Verbatim}
\end{tcolorbox}

    \hypertarget{create-correlation-plot-to-check-if-pca-is-worth-applying}{%
\subsubsection{Create correlation plot to check if PCA is worth
applying}\label{create-correlation-plot-to-check-if-pca-is-worth-applying}}

    \begin{tcolorbox}[breakable, size=fbox, boxrule=1pt, pad at break*=1mm,colback=cellbackground, colframe=cellborder]
\prompt{In}{incolor}{5}{\boxspacing}
\begin{Verbatim}[commandchars=\\\{\}]
\PY{n+nf}{options}\PY{p}{(}\PY{n}{repr.plot.width}\PY{o}{=}\PY{l+m}{8}\PY{p}{,} \PY{n}{repr.plot.height}\PY{o}{=}\PY{l+m}{8}\PY{p}{)}
\PY{c+c1}{\PYZsh{} set income related variables to the end, in order to improve on }
\PY{c+c1}{\PYZsh{} the visual aspect}
\PY{n}{subdata} \PY{o}{=} \PY{n}{subdata} \PY{o}{\PYZpc{}\PYZgt{}\PYZpc{}} \PY{n+nf}{relocate}\PY{p}{(}\PY{l+s}{\PYZdq{}}\PY{l+s}{ShareWomen\PYZdq{}}\PY{p}{,} \PY{n}{.after} \PY{o}{=} \PY{n+nf}{last\PYZus{}col}\PY{p}{(}\PY{p}{)}\PY{p}{)} \PY{o}{\PYZpc{}\PYZgt{}\PYZpc{}}
    \PY{n+nf}{relocate}\PY{p}{(}\PY{l+s}{\PYZdq{}}\PY{l+s}{Median\PYZdq{}}\PY{p}{,} \PY{n}{.after} \PY{o}{=} \PY{n+nf}{last\PYZus{}col}\PY{p}{(}\PY{p}{)}\PY{p}{)} \PY{o}{\PYZpc{}\PYZgt{}\PYZpc{}}
    \PY{n+nf}{relocate}\PY{p}{(}\PY{l+s}{\PYZdq{}}\PY{l+s}{P25th\PYZdq{}}\PY{p}{,} \PY{n}{.after} \PY{o}{=} \PY{n+nf}{last\PYZus{}col}\PY{p}{(}\PY{p}{)}\PY{p}{)} \PY{o}{\PYZpc{}\PYZgt{}\PYZpc{}}
    \PY{n+nf}{relocate}\PY{p}{(}\PY{l+s}{\PYZdq{}}\PY{l+s}{P75th\PYZdq{}}\PY{p}{,} \PY{n}{.after} \PY{o}{=} \PY{n+nf}{last\PYZus{}col}\PY{p}{(}\PY{p}{)}\PY{p}{)} \PY{o}{\PYZpc{}\PYZgt{}\PYZpc{}}
    \PY{n+nf}{relocate}\PY{p}{(}\PY{l+s}{\PYZdq{}}\PY{l+s}{Unemployment\PYZus{}rate\PYZdq{}}\PY{p}{,} \PY{n}{.after} \PY{o}{=} \PY{n+nf}{last\PYZus{}col}\PY{p}{(}\PY{p}{)}\PY{p}{)}
\PY{n+nf}{corrplot}\PY{p}{(}\PY{n+nf}{cor}\PY{p}{(}\PY{n}{subdata}\PY{p}{)}\PY{p}{,} \PY{n}{method}\PY{o}{=}\PY{l+s}{\PYZdq{}}\PY{l+s}{circle\PYZdq{}}\PY{p}{,} \PY{n}{tl.col} \PY{o}{=} \PY{l+s}{\PYZdq{}}\PY{l+s}{black\PYZdq{}}\PY{p}{)}
\end{Verbatim}
\end{tcolorbox}

    \begin{center}
    \adjustimage{max size={0.9\linewidth}{0.9\paperheight}}{output_6_0.png}
    \end{center}
    { \hspace*{\fill} \\}
    
    Groups of highly correlated variables that will be suitable for
dimensionality reduction. Some of the existing features are computed
from others. In this case the high correlation makes sense, but others,
such as Share of women and median income have a negative correlation
mostly as a result of socio-economic factors rather than feature
engineering.

    \begin{tcolorbox}[breakable, size=fbox, boxrule=1pt, pad at break*=1mm,colback=cellbackground, colframe=cellborder]
\prompt{In}{incolor}{6}{\boxspacing}
\begin{Verbatim}[commandchars=\\\{\}]
\PY{c+c1}{\PYZsh{} Apply pca to the data and specify that the features should}
\PY{c+c1}{\PYZsh{} be scaled and centered}
\PY{n}{pca} \PY{o}{=} \PY{n+nf}{prcomp}\PY{p}{(}\PY{n}{subdata}\PY{p}{,} \PY{n}{scale}\PY{o}{=}\PY{k+kc}{TRUE}\PY{p}{,} \PY{n}{center}\PY{o}{=}\PY{k+kc}{TRUE}\PY{p}{)}
\end{Verbatim}
\end{tcolorbox}

    \hypertarget{variance-explained-by-the-first-3-principal-components}{%
\subsubsection{Variance explained by the first 3 principal
components}\label{variance-explained-by-the-first-3-principal-components}}

    \begin{tcolorbox}[breakable, size=fbox, boxrule=1pt, pad at break*=1mm,colback=cellbackground, colframe=cellborder]
\prompt{In}{incolor}{7}{\boxspacing}
\begin{Verbatim}[commandchars=\\\{\}]
\PY{n}{lambdas} \PY{o}{=} \PY{n}{pca}\PY{o}{\PYZdl{}}\PY{n}{sdev}\PY{o}{\PYZca{}}\PY{l+m}{2}
\PY{n+nf}{print}\PY{p}{(}\PY{n+nf}{paste}\PY{p}{(}\PY{l+s}{\PYZdq{}}\PY{l+s}{Variance explained by first 2 components\PYZdq{}}\PY{p}{,} \PY{n+nf}{round}\PY{p}{(}\PY{n+nf}{sum}\PY{p}{(}\PY{n}{lambdas}\PY{p}{[}\PY{l+m}{1}\PY{o}{:}\PY{l+m}{2}\PY{p}{]}\PY{p}{)}\PY{o}{/}\PY{n+nf}{sum}\PY{p}{(}\PY{n}{lambdas}\PY{p}{)}\PY{p}{,} \PY{l+m}{2}\PY{p}{)}\PY{p}{)}\PY{p}{)}
\end{Verbatim}
\end{tcolorbox}

    \begin{Verbatim}[commandchars=\\\{\}]
[1] "Variance explained by first 2 components 0.83"
    \end{Verbatim}

    \begin{tcolorbox}[breakable, size=fbox, boxrule=1pt, pad at break*=1mm,colback=cellbackground, colframe=cellborder]
\prompt{In}{incolor}{8}{\boxspacing}
\begin{Verbatim}[commandchars=\\\{\}]
\PY{c+c1}{\PYZsh{} Plot screeplot to help select how many PCAs to keep}
\PY{n+nf}{options}\PY{p}{(}\PY{n}{repr.plot.width}\PY{o}{=}\PY{l+m}{10}\PY{p}{,} \PY{n}{repr.plot.height}\PY{o}{=}\PY{l+m}{8}\PY{p}{)}
\PY{n+nf}{fviz\PYZus{}eig}\PY{p}{(}\PY{n}{pca}\PY{p}{)} \PY{o}{+} \PY{n+nf}{theme}\PY{p}{(}
    \PY{n}{plot.title} \PY{o}{=} \PY{n+nf}{element\PYZus{}text}\PY{p}{(}\PY{n}{hjust} \PY{o}{=} \PY{l+m}{0.5}\PY{p}{)}\PY{p}{,}
    \PY{n}{text} \PY{o}{=} \PY{n+nf}{element\PYZus{}text}\PY{p}{(}\PY{n}{size} \PY{o}{=} \PY{l+m}{20}\PY{p}{)}
\PY{p}{)}
\end{Verbatim}
\end{tcolorbox}

    \begin{center}
    \adjustimage{max size={0.9\linewidth}{0.9\paperheight}}{output_11_0.png}
    \end{center}
    { \hspace*{\fill} \\}
    
    This suggests that first 2 PCA components should probably be kept. Third
one has an extra 6\% of explained variance. Some additional analysis can
be done to verify if clustering has more informative results with 3 PCAs
instead of 2.

    \hypertarget{plot-first-two-pcas-and-existing-features}{%
\subsubsection{Plot first two PCAs and existing
features}\label{plot-first-two-pcas-and-existing-features}}

    \begin{tcolorbox}[breakable, size=fbox, boxrule=1pt, pad at break*=1mm,colback=cellbackground, colframe=cellborder]
\prompt{In}{incolor}{9}{\boxspacing}
\begin{Verbatim}[commandchars=\\\{\}]
\PY{n+nf}{options}\PY{p}{(}\PY{n}{repr.plot.width}\PY{o}{=}\PY{l+m}{10}\PY{p}{,} \PY{n}{repr.plot.height}\PY{o}{=}\PY{l+m}{7}\PY{p}{)}
\PY{n}{factoextra}\PY{o}{::}\PY{n+nf}{fviz\PYZus{}pca\PYZus{}var}\PY{p}{(}\PY{n}{pca}\PY{p}{,}              
             \PY{n}{col.var} \PY{o}{=} \PY{l+s}{\PYZdq{}}\PY{l+s}{contrib\PYZdq{}}\PY{p}{,} \PY{c+c1}{\PYZsh{} Color by contributions to the PC}
             \PY{n}{gradient.cols} \PY{o}{=} \PY{n+nf}{c}\PY{p}{(}\PY{l+s}{\PYZdq{}}\PY{l+s}{\PYZsh{}00AFBB\PYZdq{}}\PY{p}{,} \PY{l+s}{\PYZdq{}}\PY{l+s}{\PYZsh{}E7B800\PYZdq{}}\PY{p}{,} \PY{l+s}{\PYZdq{}}\PY{l+s}{\PYZsh{}FC4E07\PYZdq{}}\PY{p}{)}\PY{p}{,}
             \PY{n}{repel} \PY{o}{=} \PY{k+kc}{TRUE}\PY{p}{,}
             \PY{n}{title} \PY{o}{=} \PY{l+s}{\PYZdq{}}\PY{l+s}{PCA1 vs PCA2\PYZdq{}}\PY{p}{)} \PY{o}{+} \PY{n+nf}{theme}\PY{p}{(}
    \PY{n}{plot.title} \PY{o}{=} \PY{n+nf}{element\PYZus{}text}\PY{p}{(}\PY{n}{hjust} \PY{o}{=} \PY{l+m}{0.5}\PY{p}{)}\PY{p}{,}
    \PY{n}{text} \PY{o}{=} \PY{n+nf}{element\PYZus{}text}\PY{p}{(}\PY{n}{size} \PY{o}{=} \PY{l+m}{16}\PY{p}{)}
\PY{p}{)}
\end{Verbatim}
\end{tcolorbox}

    \begin{Verbatim}[commandchars=\\\{\}]
Warning message:
“ggrepel: 7 unlabeled data points (too many overlaps). Consider increasing
max.overlaps”
    \end{Verbatim}

    \begin{center}
    \adjustimage{max size={0.9\linewidth}{0.9\paperheight}}{output_14_1.png}
    \end{center}
    { \hspace*{\fill} \\}
    
    \begin{tcolorbox}[breakable, size=fbox, boxrule=1pt, pad at break*=1mm,colback=cellbackground, colframe=cellborder]
\prompt{In}{incolor}{10}{\boxspacing}
\begin{Verbatim}[commandchars=\\\{\}]
\PY{c+c1}{\PYZsh{} Function to compute the Gini Index of a cluster}
\PY{n}{get\PYZus{}gini} \PY{o}{=} \PY{n+nf}{function}\PY{p}{(}\PY{n}{clusters}\PY{p}{)} \PY{p}{\PYZob{}}
    \PY{n}{grouped} \PY{o}{=} \PY{n}{clusters} \PY{o}{\PYZpc{}\PYZgt{}\PYZpc{}} \PY{n+nf}{group\PYZus{}by}\PY{p}{(}\PY{n}{category}\PY{p}{)} \PY{o}{\PYZpc{}\PYZgt{}\PYZpc{}} \PY{n+nf}{count}\PY{p}{(}\PY{p}{)}
    \PY{n}{grouped}\PY{p}{[}\PY{l+s}{\PYZdq{}}\PY{l+s}{percentage\PYZdq{}}\PY{p}{]} \PY{o}{=} \PY{n}{grouped}\PY{p}{[}\PY{l+s}{\PYZdq{}}\PY{l+s}{n\PYZdq{}}\PY{p}{]} \PY{o}{/} \PY{n+nf}{sum}\PY{p}{(}\PY{n}{grouped}\PY{p}{[}\PY{l+s}{\PYZdq{}}\PY{l+s}{n\PYZdq{}}\PY{p}{]}\PY{p}{)}
    
    \PY{n+nf}{return}\PY{p}{(}\PY{n+nf}{sum}\PY{p}{(}\PY{n}{grouped}\PY{p}{[}\PY{l+s}{\PYZdq{}}\PY{l+s}{percentage\PYZdq{}}\PY{p}{]}  \PY{o}{*} \PY{p}{(}\PY{l+m}{1} \PY{o}{\PYZhy{}} \PY{n}{grouped}\PY{p}{[}\PY{l+s}{\PYZdq{}}\PY{l+s}{percentage\PYZdq{}}\PY{p}{]}\PY{p}{)}\PY{p}{)}\PY{p}{)}
\PY{p}{\PYZcb{}}
\end{Verbatim}
\end{tcolorbox}

    \begin{tcolorbox}[breakable, size=fbox, boxrule=1pt, pad at break*=1mm,colback=cellbackground, colframe=cellborder]
\prompt{In}{incolor}{11}{\boxspacing}
\begin{Verbatim}[commandchars=\\\{\}]
\PY{c+c1}{\PYZsh{} This is my St Andrews ID, use it for reproductibility}
\PY{n+nf}{set.seed}\PY{p}{(}\PY{l+m}{210001411}\PY{p}{)}

\PY{c+c1}{\PYZsh{} Store best configuration for the clusters}
\PY{n}{best\PYZus{}gini} \PY{o}{=} \PY{l+m}{1}
\PY{n}{best\PYZus{}configuration} \PY{o}{=} \PY{l+m}{\PYZhy{}1}
\PY{n}{best\PYZus{}km} \PY{o}{=} \PY{k+kc}{NA}
\PY{n}{ginies\PYZus{}per\PYZus{}cluster} \PY{o}{=} \PY{n+nf}{c}\PY{p}{(}\PY{p}{)}
\PY{n}{best\PYZus{}clusters} \PY{o}{=} \PY{k+kc}{NA}

\PY{c+c1}{\PYZsh{} For each number of clusters selected}
\PY{c+c1}{\PYZsh{} Calculate the gini index of each cluster }
\PY{c+c1}{\PYZsh{} and find the mean gini index for a particular}
\PY{c+c1}{\PYZsh{} number of clusters between all clusters}

\PY{n+nf}{for}\PY{p}{(}\PY{n}{nr\PYZus{}clusters} \PY{n}{in} \PY{n+nf}{seq}\PY{p}{(}\PY{l+m}{2}\PY{p}{,} \PY{l+m}{16}\PY{p}{)}\PY{p}{)} \PY{p}{\PYZob{}}
    \PY{n}{res.km} \PY{o}{=} \PY{n+nf}{kmeans}\PY{p}{(}\PY{n}{pca}\PY{o}{\PYZdl{}}\PY{n}{x}\PY{p}{[}\PY{l+m}{1}\PY{o}{:}\PY{l+m}{172}\PY{p}{,} \PY{l+m}{1}\PY{o}{:}\PY{l+m}{2}\PY{p}{]}\PY{p}{,} \PY{n}{nr\PYZus{}clusters}\PY{p}{,} \PY{n}{nstart}\PY{o}{=}\PY{l+m}{20}\PY{p}{,} \PY{n}{iter.max}\PY{o}{=}\PY{l+m}{500}\PY{p}{)}
    \PY{n}{clusters} \PY{o}{=} \PY{n+nf}{data.frame}\PY{p}{(}\PY{n}{cluster}\PY{o}{=}\PY{n}{res.km}\PY{o}{\PYZdl{}}\PY{n}{cluster}\PY{p}{,} \PY{n}{major}\PY{o}{=}\PY{n}{data}\PY{o}{\PYZdl{}}\PY{n}{Major}\PY{p}{,} \PY{n}{category} \PY{o}{=} \PY{n}{data}\PY{o}{\PYZdl{}}\PY{n}{Major\PYZus{}category}\PY{p}{)}
    \PY{n}{clusters} \PY{o}{=} \PY{n}{clusters}\PY{p}{[}\PY{n+nf}{order}\PY{p}{(}\PY{n}{clusters}\PY{o}{\PYZdl{}}\PY{n}{category}\PY{p}{)}\PY{p}{,} \PY{p}{]}
    
    \PY{n}{ginies} \PY{o}{=} \PY{n+nf}{c}\PY{p}{(}\PY{p}{)}
    \PY{n}{total\PYZus{}gini} \PY{o}{=} \PY{l+m}{0}
    
    \PY{n+nf}{for}\PY{p}{(}\PY{n}{i} \PY{n}{in} \PY{n+nf}{seq}\PY{p}{(}\PY{l+m}{1}\PY{o}{:}\PY{n}{nr\PYZus{}clusters}\PY{p}{)}\PY{p}{)} \PY{p}{\PYZob{}}
        \PY{c+c1}{\PYZsh{} Calculate gini index for each cluster}
        \PY{n}{cluster\PYZus{}gini} \PY{o}{=} \PY{n+nf}{get\PYZus{}gini}\PY{p}{(}\PY{n}{clusters}\PY{p}{[}\PY{n}{clusters}\PY{p}{[}\PY{l+s}{\PYZdq{}}\PY{l+s}{cluster\PYZdq{}}\PY{p}{]} \PY{o}{==} \PY{n}{i}\PY{p}{,} \PY{p}{]}\PY{p}{)}
        \PY{n}{ginies} \PY{o}{=} \PY{n+nf}{c}\PY{p}{(}\PY{n}{ginies}\PY{p}{,} \PY{n}{cluster\PYZus{}gini}\PY{p}{)}
        \PY{n}{total\PYZus{}gini} \PY{o}{=} \PY{n}{total\PYZus{}gini} \PY{o}{+} \PY{n}{cluster\PYZus{}gini}
    \PY{p}{\PYZcb{}}
        
    \PY{n}{mean\PYZus{}gini} \PY{o}{=} \PY{n}{total\PYZus{}gini}\PY{o}{/}\PY{n}{nr\PYZus{}clusters}
    \PY{n+nf}{if}\PY{p}{(}\PY{n}{mean\PYZus{}gini} \PY{o}{\PYZlt{}} \PY{n}{best\PYZus{}gini} \PY{o}{\PYZhy{}} \PY{l+m}{0.05}\PY{p}{)} \PY{p}{\PYZob{}}
        \PY{c+c1}{\PYZsh{} If this gini is significantly improving the }
        \PY{c+c1}{\PYZsh{} best configuration so far, store it.}
        \PY{c+c1}{\PYZsh{} If the improvement is not large enough, }
        \PY{c+c1}{\PYZsh{} avoid storing a very high number of clusters}
        
        \PY{n}{best\PYZus{}gini} \PY{o}{=} \PY{n}{mean\PYZus{}gini}
        \PY{n}{best\PYZus{}configuration} \PY{o}{=} \PY{n}{nr\PYZus{}clusters}
        \PY{n}{best\PYZus{}km} \PY{o}{=} \PY{n}{res.km}
        \PY{n}{ginies\PYZus{}per\PYZus{}cluster} \PY{o}{=} \PY{n}{ginies}
        \PY{n}{best\PYZus{}clusters} \PY{o}{=} \PY{n}{clusters}
    \PY{p}{\PYZcb{}}
\PY{p}{\PYZcb{}}
\PY{n+nf}{print}\PY{p}{(}\PY{n+nf}{paste}\PY{p}{(}\PY{l+s}{\PYZdq{}}\PY{l+s}{Best gini index\PYZdq{}}\PY{p}{,} \PY{n+nf}{round}\PY{p}{(}\PY{n}{best\PYZus{}gini}\PY{p}{,} \PY{l+m}{2}\PY{p}{)}\PY{p}{,} \PY{l+s}{\PYZdq{}}\PY{l+s}{and number of clusters\PYZdq{}}\PY{p}{,} \PY{n}{best\PYZus{}configuration}\PY{p}{)}\PY{p}{)}
\PY{n}{gini\PYZus{}by\PYZus{}cluster} \PY{o}{=} \PY{n+nf}{data.frame}\PY{p}{(}\PY{n}{cluster} \PY{o}{=} \PY{n+nf}{seq}\PY{p}{(}\PY{l+m}{1}\PY{p}{,} \PY{n}{best\PYZus{}configuration}\PY{p}{)}\PY{p}{,} \PY{n}{gini} \PY{o}{=} \PY{n}{ginies\PYZus{}per\PYZus{}cluster}\PY{p}{)}
\PY{n}{gini\PYZus{}by\PYZus{}cluster} \PY{o}{=} \PY{n}{gini\PYZus{}by\PYZus{}cluster}\PY{p}{[}\PY{n+nf}{order}\PY{p}{(}\PY{n}{gini\PYZus{}by\PYZus{}cluster}\PY{o}{\PYZdl{}}\PY{n}{gini}\PY{p}{)}\PY{p}{,}\PY{p}{]}
\end{Verbatim}
\end{tcolorbox}

    \begin{Verbatim}[commandchars=\\\{\}]
[1] "Best gini index 0.64 and number of clusters 9"
    \end{Verbatim}

    \begin{tcolorbox}[breakable, size=fbox, boxrule=1pt, pad at break*=1mm,colback=cellbackground, colframe=cellborder]
\prompt{In}{incolor}{13}{\boxspacing}
\begin{Verbatim}[commandchars=\\\{\}]
\PY{c+c1}{\PYZsh{} Attach a column for the cluster index in the initial dataset}
\PY{n+nf}{names}\PY{p}{(}\PY{n}{best\PYZus{}clusters}\PY{p}{)}\PY{p}{[}\PY{n+nf}{names}\PY{p}{(}\PY{n}{best\PYZus{}clusters}\PY{p}{)} \PY{o}{==} \PY{l+s}{\PYZsq{}}\PY{l+s}{major\PYZsq{}}\PY{p}{]} \PY{o}{=} \PY{l+s}{\PYZsq{}}\PY{l+s}{Major\PYZsq{}}
\PY{n}{data} \PY{o}{=} \PY{n+nf}{merge}\PY{p}{(}\PY{n}{x}\PY{o}{=}\PY{n}{data}\PY{p}{,}\PY{n}{y}\PY{o}{=}\PY{n}{best\PYZus{}clusters}\PY{p}{[}\PY{o}{\PYZhy{}}\PY{n+nf}{c}\PY{p}{(}\PY{l+m}{3}\PY{p}{)}\PY{p}{]}\PY{p}{,}\PY{n}{by}\PY{o}{=}\PY{l+s}{\PYZdq{}}\PY{l+s}{Major\PYZdq{}}\PY{p}{)}
\end{Verbatim}
\end{tcolorbox}

    \begin{tcolorbox}[breakable, size=fbox, boxrule=1pt, pad at break*=1mm,colback=cellbackground, colframe=cellborder]
\prompt{In}{incolor}{ }{\boxspacing}
\begin{Verbatim}[commandchars=\\\{\}]
\PY{c+c1}{\PYZsh{} Find summaries for each cluster (in this case the mean)}
\PY{n}{data} \PY{o}{\PYZpc{}\PYZgt{}\PYZpc{}} \PY{n+nf}{select}\PY{p}{(}\PY{o}{\PYZhy{}}\PY{n}{Major}\PY{p}{,} \PY{o}{\PYZhy{}}\PY{n}{Major\PYZus{}category}\PY{p}{)} \PY{o}{\PYZpc{}\PYZgt{}\PYZpc{}} \PY{n+nf}{group\PYZus{}by}\PY{p}{(}\PY{n}{data}\PY{o}{\PYZdl{}}\PY{n}{cluster}\PY{p}{)} \PY{o}{\PYZpc{}\PYZgt{}\PYZpc{}} \PY{n+nf}{summarise}\PY{p}{(}\PY{n+nf}{across}\PY{p}{(}\PY{n+nf}{everything}\PY{p}{(}\PY{p}{)}\PY{p}{,} \PY{n}{mean}\PY{p}{)}\PY{p}{)}
\end{Verbatim}
\end{tcolorbox}

    \hypertarget{visualise-polygons-of-clusters-in-2d-using-the-first-two-pcas}{%
\subsubsection{Visualise polygons of clusters in 2D using the first two
PCAs}\label{visualise-polygons-of-clusters-in-2d-using-the-first-two-pcas}}

Alternatively the data can be the initial datapoints and pairs of
features from it.

    \begin{tcolorbox}[breakable, size=fbox, boxrule=1pt, pad at break*=1mm,colback=cellbackground, colframe=cellborder]
\prompt{In}{incolor}{15}{\boxspacing}
\begin{Verbatim}[commandchars=\\\{\}]
\PY{n+nf}{options}\PY{p}{(}\PY{n}{repr.plot.width}\PY{o}{=}\PY{l+m}{10}\PY{p}{,} \PY{n}{repr.plot.height}\PY{o}{=}\PY{l+m}{8}\PY{p}{)}
\PY{n+nf}{fviz\PYZus{}cluster}\PY{p}{(}\PY{n}{best\PYZus{}km}\PY{p}{,} \PY{n}{data} \PY{o}{=} \PY{n}{pca}\PY{o}{\PYZdl{}}\PY{n}{x}\PY{p}{[}\PY{l+m}{1}\PY{o}{:}\PY{l+m}{172}\PY{p}{,} \PY{l+m}{1}\PY{o}{:}\PY{l+m}{2}\PY{p}{]}\PY{p}{,}
             \PY{n}{palette} \PY{o}{=} \PY{n}{palette}\PY{p}{,} 
             \PY{n}{geom} \PY{o}{=} \PY{l+s}{\PYZdq{}}\PY{l+s}{point\PYZdq{}}\PY{p}{,}
             \PY{n}{ellipse.type} \PY{o}{=} \PY{l+s}{\PYZdq{}}\PY{l+s}{convex\PYZdq{}}\PY{p}{,} 
             \PY{n}{ggtheme} \PY{o}{=} \PY{n+nf}{theme\PYZus{}bw}\PY{p}{(}\PY{p}{)}
             \PY{p}{)} \PY{o}{+} \PY{n+nf}{theme}\PY{p}{(}
    \PY{n}{plot.title} \PY{o}{=} \PY{n+nf}{element\PYZus{}text}\PY{p}{(}\PY{n}{hjust} \PY{o}{=} \PY{l+m}{0.5}\PY{p}{)}\PY{p}{,}
    \PY{n}{text} \PY{o}{=} \PY{n+nf}{element\PYZus{}text}\PY{p}{(}\PY{n}{size} \PY{o}{=} \PY{l+m}{16}\PY{p}{)}\PY{p}{,}
    \PY{n}{legend.position}\PY{o}{=}\PY{l+s}{\PYZdq{}}\PY{l+s}{none\PYZdq{}}
\PY{p}{)}
\end{Verbatim}
\end{tcolorbox}

    \begin{center}
    \adjustimage{max size={0.9\linewidth}{0.9\paperheight}}{output_20_0.png}
    \end{center}
    { \hspace*{\fill} \\}
    
    \hypertarget{find-if-number-of-clusters-can-be-picked-using-sillhouette-plots}{%
\subsubsection{Find if number of clusters can be picked using
sillhouette
plots}\label{find-if-number-of-clusters-can-be-picked-using-sillhouette-plots}}

    \begin{tcolorbox}[breakable, size=fbox, boxrule=1pt, pad at break*=1mm,colback=cellbackground, colframe=cellborder]
\prompt{In}{incolor}{16}{\boxspacing}
\begin{Verbatim}[commandchars=\\\{\}]
\PY{n}{sil} \PY{o}{\PYZlt{}\PYZhy{}} \PY{n+nf}{silhouette}\PY{p}{(}\PY{n}{x} \PY{o}{=} \PY{n}{best\PYZus{}km}\PY{o}{\PYZdl{}}\PY{n}{cluster}\PY{p}{,} \PY{n}{dist} \PY{o}{=} \PY{n+nf}{dist}\PY{p}{(}\PY{n}{pca}\PY{o}{\PYZdl{}}\PY{n}{x}\PY{p}{[}\PY{l+m}{1}\PY{o}{:}\PY{l+m}{172}\PY{p}{,} \PY{l+m}{1}\PY{o}{:}\PY{l+m}{2}\PY{p}{]}\PY{p}{)}\PY{p}{)}
\PY{n+nf}{fviz\PYZus{}silhouette}\PY{p}{(}\PY{n}{sil}\PY{p}{)} \PY{o}{+}
\PY{n+nf}{scale\PYZus{}fill\PYZus{}manual}\PY{p}{(}\PY{n}{values} \PY{o}{=} \PY{n}{palette}\PY{p}{)} \PY{o}{+}
\PY{n+nf}{scale\PYZus{}color\PYZus{}manual}\PY{p}{(}\PY{n}{values} \PY{o}{=} \PY{n}{palette}\PY{p}{)}
\end{Verbatim}
\end{tcolorbox}

    \begin{Verbatim}[commandchars=\\\{\}]
  cluster size ave.sil.width
1       1   29          0.43
2       2   11          0.30
3       3    6          0.27
4       4   18          0.49
5       5    2          0.49
6       6   48          0.39
7       7   48          0.46
8       8    6          0.55
9       9    4          0.27
    \end{Verbatim}

    \begin{center}
    \adjustimage{max size={0.9\linewidth}{0.9\paperheight}}{output_22_1.png}
    \end{center}
    { \hspace*{\fill} \\}
    
    \hypertarget{does-gap-metric-suggest-a-better-number-of-clusters}{%
\subsubsection{Does gap metric suggest a better number of clusters
?}\label{does-gap-metric-suggest-a-better-number-of-clusters}}

    \begin{tcolorbox}[breakable, size=fbox, boxrule=1pt, pad at break*=1mm,colback=cellbackground, colframe=cellborder]
\prompt{In}{incolor}{18}{\boxspacing}
\begin{Verbatim}[commandchars=\\\{\}]
\PY{n+nf}{fviz\PYZus{}nbclust}\PY{p}{(}\PY{n}{x} \PY{o}{=} \PY{n}{pca}\PY{o}{\PYZdl{}}\PY{n}{x}\PY{p}{[}\PY{l+m}{1}\PY{o}{:}\PY{l+m}{172}\PY{p}{,} \PY{l+m}{1}\PY{o}{:}\PY{l+m}{2}\PY{p}{]}\PY{p}{,} \PY{n}{FUNcluster} \PY{o}{=} \PY{n}{kmeans}\PY{p}{,} \PY{n}{method} \PY{o}{=} \PY{l+s}{\PYZdq{}}\PY{l+s}{gap\PYZdq{}}\PY{p}{,} \PY{n}{k.max} \PY{o}{=} \PY{l+m}{20}\PY{p}{)}
\end{Verbatim}
\end{tcolorbox}

    \begin{center}
    \adjustimage{max size={0.9\linewidth}{0.9\paperheight}}{output_24_0.png}
    \end{center}
    { \hspace*{\fill} \\}
    

    % Add a bibliography block to the postdoc
    
    
    
\end{document}
